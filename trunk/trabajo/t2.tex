% Clase
\documentclass[11pt,a4paper,spanish,twoside]{report}

% Órdenes auxiliares
\input{inc/includes.tex}

% Encabezado y pie de página
\encabezado

\setcounter{secnumdepth}{3}
\setcounter{tocdepth}{3}

\begin{document}

% Silabación extra
\input{inc/hyphenations.tex}

% Portada
\portada{Automatización Industrial}
{Trabajo:}{Automatización del proceso de embotellado de zumos en\\ J. García Carrión(Daimiel)}
{Sergio de la Rubia García-Carpintero\\Miguel Millán Sánchez-Grande}{30 de Abril de 2010}

% Licencia
\licencia{Sergio de la Rubia García-Carpintero, Miguel Millán Sánchez-Grande}

% Índices
\tableofcontents
% \listoffigures
% \listoftables

%% INICIO DEL DOCUMENTO %%%%%%%%%%%%%%%%%%%%%%%%%%%%%%%%%%%%%%%%%%%%%%%%%

\chapter{Introducción}
%soltar el rollo de xq la hemos elegido, en que consiste.
A la hora de elegir una empresa sobre la que realizar el trabajo, se buscaron
opciones cercanas geográficamente y de fácil acceso en nuestro entorno. Por
este motivo la opción elegida fue la de la empresa J.Garcia Ca\-rrión (a partir
de ahora JGC), ubicada en Daimiel; por tratarse de una instalación próxima 
a Ciudad Real y porque el hermano de uno de los componentes de nuestro grupo
trabaja en dicha planta.

JGC es el encargado de todos los productos Don Simon. 

JGC para desarrollar su plan estratégico cuenta con disitintas bodegas y plantas
de proceso y envasado de última tecnología en distintas partes de la
geografía española, siendo una empresa pionera tanto a nivel nacional como
internacional:
\begin{itemize}
\item Jumilla (Murcia), origen de la empresa, con una capacidad de 400
  millones de litros. 
\item Daimiel (Ciudad Real), pionera en tecnología a nivel mundial, cuya alta
  producción de 800 millones de envases por año, la hace la más eficiente de
  Europa. Sobre la que se centra el trabajo.  
\item Gador (Almería), inaugurada en el año 2003, es una Planta especializada
  en tratamiento de vegetales (gazpacho, cremas de verduras naturales y
  caldos), con una capacidad de 120 millones de Kilos/año. La tecnología de
  esta planta se ha proyectado con Investigación y Desarrollo propio, único
  en la elaboración de productos naturales vegetales. 
\item Huelva, planta única en el mundo por su agricultura integrada. 
\end{itemize}

\section{Historia breve}
La familia JGC siempre a estado ligada a los viñedos y a la tradición
agrícola, tiene sus orígenes en el pueblo murciano de Jumilla. En 1890 la 
familia construyó una nueva bodega, con ciertas dimensiones para aquella
época, debido al gran auge de la exportación del vino de Jumilla a
Francia, esta exportación fue originada por un parásito llamado filoxera que
arrasó con los viñedos del país galo. Por tanto se toma el año 1890 como él de la
fundación de GARCIA-CARRIÓN.

La marca GARCIA-CARRIÓN, prosigue su andadura, creciendo poco a poco y
sobreponiéndose a adversidades como la guerra civil. Pero aún así comenzó a
distribuirse por toda España. Al continuar aumentando la demanda, se decide
construir una nueva bodega en las proximidades de Jumilla, e instalar el
primer tren de embotellado de alta capacidad.
 
Si por algo se ha caracterizado siempre la compañia a sido por la innovación
y por arriesgarse a realizar cosas que otros pensaban imposibles, de
esta forma, a principios de los 80, con la implantación de las grandes
superficies en España, el envase más utilizado para el vino de mesa era la
botella de 1L. retornable, lo que exigía la posesión varias plantas de
envasado distribuidas por toda España para atender la demanda nacional. Con
el fin de buscar un envase no retornable, práctico, económico y de poco peso
la compañía optó por la tecnología brik.

El lanzamiento del nuevo envase coincidió con el primer anuncio en televisión
de la compañía, que será siempre recordado por la frase : ``Voy a comer con
Don Simón''.

\chapter{Objetivo y motivación del sistema}

\chapter{Descripción del proceso}
El proceso que se lleva a cabo en la fábrica de JGC situada en Daimiel y
sobre el cual se centra el esfuerzo de este trabajo consiste en embotellar
zumos, sangría o incluso tinto de verano. Es decir cualquier producto de Don
Simón que esté envasado en botella de plástico.
La figura \ref{proceso} representa un mapa de todo el proceso seguido.

\imagen{proceso.pdf}{12.5}{Proceso de embotellamiento}{proceso}
\chapter{Explicación de las islas de automatización}
\section{Sopladora}
\section{Llenadora/Taponadora}
\subsection{Inspector del nivel del tapón}
\section{Etiquetadora}
\subsection{Inspector de etiquetas}
\section{Agrupadora}
\section{Paletizadora}
\section{Enfardadora}
\chapter{Beneficios de la automatización}
Introducir nuevas tecnologías en una empresa tradicional como JGC siempre
resulta un proceso complicado y que no agrada a todo el mundo. Pero no se
puede negar que supuso un empujón muy importante tanto a nivel económico como
a nivel de producción para la empresa. Errónamente se asocia el concepto de
automatización con una solución para reducir la cantidad de empleados en
plantilla, bien esta empresa es un ejemplo de que esto no sucede, ya que al
mecanizar el proceso relacionado con el empaquetado y relleno de botellas
ayudo a la expansión de la empresa creando más puesto de trabajos, eso sí
distintos a los que había con anterioridad. Aparte del beneficio económico de
reducir el número de empleados de ese área hubo otros beneficios adicionales
mayores:
\begin{description}
\item \textbf{Aumento de la eficiencia}

Los costos de producción se redujeron drásticamente al aumentar las
unidades de producto fabricadas por unidad de tiempo. 

\item \textbf{Incremento del volumen de producción}

Al aumentar el número de unidades fabricadas por unidad de tiempo se aumentó
la cantidad de unidades producidas así como el número de clientes a los que
poder atender.

\item \textbf{Estandarización de los procesos}

De esta forma se logró que los productos tuvieran siempre las mismas
características, al hacer el proceso repetitivo y siguiendo los mismos
pasos. Se puede tener la certeza que dos vasos de vino o zumo Don simon van a
tener siempre el mismo sabor, color, desidad, etc.

\item \textbf{Reducción de los problemas de calidad}

Consecuentemente con la estandarización de procesos se consiguió aumentar la
calidad, se eliminó cualquier error posible relacionado con un despiste
humano, ya fuera debido al cansanco o a una negligencia. 

\item \textbf{Aumento de la competitividad}

Todas estas mejoras, más y mejor producto en menos cantidad de
tiempo, tienen una clara repercusión en la competitividad de la empresa,
haciendo más fácil el cubrir diversas áreas de comercio. 

\item \textbf{Centralización de producción}

JGC es una de las empresas más grandes e importantes de españa en el sector,
y como se ha hablado en la introducción consta solo de cuatro fabricas. Esto
es debido a la gran productividad asociada al uso de la automatización, no es
necesaria la construcción de mas sedes para cubrir las necesida

\end{description}

Como demostración práctica de la utilidad de la automatización, la empresa
esta haciendo frente a la crisis que experimentamos en estos días con un
rebaja considerable en los precios de sus productos. Para realizar esto, JGC,
se está preparando para multiplicar la capacidad de producción 1,5 veces en
2014. Con el incremento de la producción pretende atender la demanda en
mercados internacionales para elevar su porcentaje de ventas al exterior
desde el 35\% actual al 60\%. Aún con la crisis la empresa tenía previsto
cerrar el ejercicio en curso con un beneficio antes de impuestos 22 millones
de euros, lo que supone un incremento del 22\% respecto al año anterior, y
elevar sus ventas un 12\%, hasta alcanzar los 650 millones de euros, en un
año en el que ha recortado sus precios un 20\% de media para hacer frente al
descenso provocado por la anteriormente nomberda crisis económica.
\bibliographystyle{plain} 
\bibliography{t2}

\end{document}
