% Clase
\documentclass[11pt,a4paper,spanish,twoside]{report}

% Órdenes auxiliares
\input{inc/includes.tex}

% Encabezado y pie de página
\encabezado

\setcounter{secnumdepth}{3}
\setcounter{tocdepth}{3}

\begin{document}

% Silabación extra
\input{inc/hyphenations.tex}

% Portada
\portada{Automatización Industrial}
{Trabajo:}{Automatización del proceso de embotellado de zumos en\\ J. García Carrión(Daimiel)}
{Sergio de la Rubia García-Carpintero\\Miguel Millán Sánchez-Grande}{30 de Abril de 2010}

% Licencia
\licencia{Sergio de la Rubia García-Carpintero, Miguel Millán Sánchez-Grande}

% Índices
\tableofcontents
% \listoffigures
% \listoftables

%% INICIO DEL DOCUMENTO %%%%%%%%%%%%%%%%%%%%%%%%%%%%%%%%%%%%%%%%%%%%%%%%%

\chapter{Introducción}
%soltar el rollo de xq la hemos elegido, en que consiste.
A la hora de elegir una empresa sobre la que realizar el trabajo, se buscaron
opciones cercanas geográficamente y de fácil acceso en nuestro entorno. Por
este motivo la opción elegida fue la de la empresa J.Garcia Ca\-rrión (a partir
de ahora JGC), ubicada en Daimiel; por tratarse de una instalación próxima 
a Ciudad Real y porque el hermano de uno de los componentes de nuestro grupo
trabaja en dicha planta.

JGC es el encargado de todos los productos Don Simon. 

JGC para desarrollar su plan estratégico cuenta con disitintas bodegas y plantas
de proceso y envasado de última tecnología en distintas partes de la
geografía española, siendo una empresa pionera tanto a nivel nacional como
internacional:
\begin{itemize}
\item Jumilla (Murcia), origen de la empresa, con una capacidad de 400
  millones de litros. 
\item Daimiel (Ciudad Real), pionera en tecnología a nivel mundial, cuya alta
  producción de 800 millones de envases por año, la hace la más eficiente de
  Europa. Sobre la que se centra el trabajo.  
\item Gador (Almería), inaugurada en el año 2003, es una Planta especializada
  en tratamiento de vegetales (gazpacho, cremas de verduras naturales y
  caldos), con una capacidad de 120 millones de Kilos/año. La tecnología de
  esta planta se ha proyectado con Investigación y Desarrollo propio, único
  en la elaboración de productos naturales vegetales. 
\item Huelva, planta única en el mundo por su agricultura integrada. 
\end{itemize}

\section{Historia breve}
La familia JGC siempre a estado ligada a los viñedos y a la tradición
agrícola, tiene sus orígenes en el pueblo murciano de Jumilla. En 1890 la 
familia construyó una nueva bodega, con ciertas dimensiones para aquella
época, debido al gran auge de la exportación del vino de Jumilla a
Francia, esta exportación fue originada por un parásito llamado filoxera que
arrasó con los viñedos del país galo. Por tanto se toma el año 1890 como él de la
fundación de GARCIA-CARRIÓN.

La marca GARCIA-CARRIÓN, prosigue su andadura, creciendo poco a poco y
sobreponiéndose a adversidades como la guerra civil. Pero aún así comenzó a
distribuirse por toda España. Al continuar aumentando la demanda, se decide
construir una nueva bodega en las proximidades de Jumilla, e instalar el
primer tren de embotellado de alta capacidad.
 
Si por algo se ha caracterizado siempre la compañia a sido por la innovación
y por arriesgarse a realizar cosas que otros pensaban imposibles, de
esta forma, a principios de los 80, con la implantación de las grandes
superficies en España, el envase más utilizado para el vino de mesa era la
botella de 1L. retornable, lo que exigía la posesión varias plantas de
envasado distribuidas por toda España para atender la demanda nacional. Con
el fin de buscar un envase no retornable, práctico, económico y de poco peso
la compañía optó por la tecnología brik.

El lanzamiento del nuevo envase coincidió con el primer anuncio en televisión
de la compañía, que será siempre recordado por la frase : ``Voy a comer con
Don Simón''.

\chapter{Objetivo y motivación del sistema}

\chapter{Descripción del proceso}
\chapter{Explicación de las islas de automatización}
\section{Sopladora}
\section{Llenadora/Taponadora}
\subsection{Inspector del nivel del tapón}
\section{Etiquetadora}
\subsection{Inspector de etiquetas}
\section{Agrupadora}
\section{Paletizadora}
\section{Enfardadora}
\chapter{Beneficios de la automatización}
La automatización es el proceso mediante el cual logramos que un equipo,
maquina o herramienta realice un trabajo que hemos estado llevando a cabo
manualmente o como parte de las funciones de un empleado (persona).
Aunque la automatización erróneamente se ve como una solución para reducir
la cantidad de empleados, la realidad es que hay otros beneficios adicionales
mayores, que normalmente no visualizamos. La implementación adecuada (como se
le presenta al empleado) es la clave para maximizar los beneficios derivados
de dicha práctica. 
Entre los principales beneficios de la automatización podemos mencionar:
\begin{itemize}
\item Aumento de la eficiencia
Los costos de producción se pueden reducir drásticamente al aumentar las
unidades de producto fabricadas en una misma unidad de tiempo. 

\item Incremento del volumen de producción
Podemos aumentar la cantidad de unidades producidas o el número de clientes
servidos, manteniendo el ritmo de trabajo consistentemente por un periodo de
tiempo mayor y sin necesidad de paradas. 

\item Estandarización de los procesos
Se logra que el producto final mantenga las mismas características y que el
proceso se repita cada vez de la misma forma. 

\item Reducción de los problemas de Calidad
Al lograr la estandarización se impacta positivamente el aspecto de calidad,
pues se reduce los errores relacionados al factor humano debido a situaciones
como cansancio, descuido, etc. 

\item Mejora del ambiente de trabajo
Si la estandarización se implementa de forma adecuada (para ayudar y
facilitar la tarea del empleado), se logra una mejora en el ambiente de
trabajo y un incremento del apoderamiento. 
\end{itemize}

La automatización es un proceso sencillo que se aplica a diferentes
escalas. La automatización puede significar algo tan simple como cambiar el
proceso de mezcla de un líquido de forma manual; al uso de un agitador simple
(motor con un aspa). Apagar o encender un equipo o las luces con un
dispositivo automático (foto celda o -Y´limitswitch¡ es otro ejemplo
sencillo. 
\bibliographystyle{plain} 
\bibliography{t2}

\end{document}
\chapter*{Introducción}
A la hora de decidir la institución sobre la cual centrar nuestra 
investigación, empezamos analizando la posibilidad de buscar una empresa 
cercana geográficamente como podría haber sido el aeropuerto de Ciudad Real. 
Pero ante la posibilidad de encontrar dificultades a la hora de recopilar 
información nos decantamos por una entidad pública. Nuestra primera opción fue 
la ESI, pero buscando, encontramos mucha más información sobre la Universidad 
de Málaga, de ahí nuestra elección.

La universidad de Málaga es una universidad pública, joven y dinámica que ha 
apostado decididamente por la calidad en la docencia, la investigación y por el
servicio al alumno. Cuenta con más de 40.000 alumnos matriculados y 2.000 
investigadores. 

Para la realización del trabajo hemos usado como ayuda el estándar IEEE
1058 \cite{ieee}, la metodología MÉTRICA 3 \cite{met} y diferentes apuntes e
informaciones recogidas a través de Internet como material de complemento. 

\chapter{Introducción al plan de gestión del proyecto}
\section{Visión general del proyecto}
Se trata de realizar una aplicación que controle todos los aspectos 
relacionados con la generación automatizada de guías docentes para la UMA.
Esta aplicación no solo se encarga de construir una guía docente a partir de 
los datos almacenados en las distintas bases de datos con las que cuenta la 
universidad, también se encarga de la adaptación de los planes de estudio 
antiguos a los nuevos acordados por el EEES, la asignación del profesorado a 
las distintas asignaturas, el establecimiento de los horarios lectivos y los 
horarios de exámenes, así como de sus localizaciones.

La implantación de la aplicación tiene como objetivo principal la agilización 
de la realización de las guías docentes. Este objetivo principal lleva 
consigo el cumplimiento de otros subobjetivos como, la actualización en tiempo 
real de los contenidos de la guías docentes, así como la mejora en la 
accesibilidad a dichos contenidos y la reducción en el gasto de la 
contratación de personal para la elaboración de dichas guías.

Para la realización del proyecto se cuenta con: Dña. Adelaida de la Calle, 
rectora de la UMA y jefa de proyecto; Dña. María Valpuesta, vicerrectora de 
Innovación y Desarrollo Tecnológico y responsable del proyecto; D. Luis 
Muñoz, coordinador; un grupo de trabajo formado por: D. Sergio de la Rubia, 
D. Miguel Millán, Dña. Alicia Serrano y D. Juan Miguel Torres; y la 
colaboración de un analista, un programador, un usuario experto, un 
secretario y un operario para el servicio técnico.

La realización del proyecto sigue un ciclo de vida en cascada, con un 
desarrollo de 88 días laborables, con objeto de que esté concluido para el 
inicio del nuevo año académico 2010-2011.

\section{Entregables del proyecto}
El conjunto de entregables estará constituido por:
\begin{itemize}
\item El software desarrollado, que no sólo incluye la posibilidad de
  consultar y estructurar el contenido de las guías docentes de cada uno de
  los estudios que se imparten en la UMA, sino que también contiene las
  funcionalidades de generación de automática de horarios, asignación
  automática de aulas y la adaptación automática entre planes de estudios. El
  software se integra en los equipos que se encargan de la gestión de la
  UMA. La guía docente resultante será consultable desde la página web de la
  universidad.
\item Una completa documentación para el soporte y mantenimiento del sistema,
  así como un manual de usuario para el personal que gestiona el sistema. 
\end{itemize}

\section{Material de referencia}
El material de referencia usado como apoyo en el proyecto ha sido el siguiente:
\begin{itemize}
\item Listado de requisitos facilitados por la universidad, con las 
  características y especificaciones que el software debe cumplir.
\item Recomendaciones de la rectora y la vicerrectora de Innovación y 
  Desarrollo Tecnológico acerca del desarrollo del plan de proyecto.
\item Estándar IEEE 1058 \cite{ieee}.
\item Modelo MÉTRICA 3 \cite{met}.
\item Documentación acerca de las redes, SSOO, metodologías, etc. con los que 
  cuenta la universidad y con los que debería contar.
\end{itemize}

\section{Definiciones y acrónimos}
\begin{description}
\item[EEES] Espacio Europeo de Educación Superior.
\item[IEEE] Instituto de Ingenieros Electricistas y Electrónicos.
\item[SI] Sistema de Información.
\item[SSOO] Sistemas Operativos.
\item[TI] Tecnología de la Información.
\item[UMA] Universidad de Málaga.
\item[WAN] Wide Area Network.
\end{description}

\chapter{Organización del proyecto}
\section{Modelo de procesos}
El ciclo de vida a seguir para este proyecto es el de ``cascada''. Dicho 
modelo marca en cada momento las acciones a realizar. Hay que puntualizar 
que si se necesita repasar alguna fase, supondría pérdidas de tiempo y dinero.
El esquema del modelo es el siguiente:

\noindent
$>>$ Análisis \\
$>>>>$ Diseño \\
$>>>>>>$ Implementación \\
$>>>>>>>>$ Pruebas

Al tratarse de un ciclo de vida en cascada, cada fase requiere que la anterior 
este completada, lo que implica que todas las actividades deben de ser 
realizadas en función al orden establecida. Al final de cada fase se produce 
como resultado de salida uno o varios entregables, según la etapa.

\section{Estructura organizativa}
El personal requerido para este proyecto organizado por su jerarquía es el 
siguiente:
\begin{itemize}
\item Jefe de proyecto, Dña. Adelaida de la Calle, rectora de la UMA.
  Responsable, Dña. María Valpuesta, vicerrectorado de Innovación y Desarrollo 
  Tecnológico.
\item Coordinador, D. Luis Muñoz Villarreal.
\item Grupo de trabajo, formado por: D. Sergio de la Rubia García-Carpintero, 
D. Miguel Millán Sánchez-Grande, Dña. Alicia Serrano Sánchez y D. Juan Miguel
Torres Triviño. 
\item A estos hay que añadir los empleados contratados: un analista, un 
  programador, un usuario experto, un secretario y un operario de servicio 
  técnico.
\end{itemize}

\section{Fronteras e interfaces organizativas}
Para realizar, probar y configurar el sistema resultante del proyecto se 
necesita una estrecha colaboración con todos los departamentos que componen 
la UMA. 

\section{Responsabilidades}
En la tabla \ref{Tab:Respon} se muestran las competencias de cada miembro del
equipo.
\begin{sidewaystable}[!h]
  \centering
  \scriptsize
  \begin{tabular}{l|c|b{1.5cm}<{\centering}|c|c|c|c|b{2.5cm}<{\centering}}
    & \textbf{Coordinador} & \textbf{Grupo de trabajo} &
    \textbf{Analista} & \textbf{Programador} & \textbf{Secretario} &
    \textbf{Usuario experto} & \textbf{Operario servicio técnico}\\
    \hline \hline
    Gestión                     & x &   & x &   &   &   &    \\ 
    \hline
    Gestión de la configuración &   & x &   &   &   &   &    \\
    \hline
    Planificación               & x & x & x &   &   &   &    \\
    \hline
    Requisitos                  &   & x & x &   &   &   &    \\
    \hline
    Diseño                      &   & x &   &   &   &   &    \\
    \hline
    Programación                &   &   &   & x &   &   &    \\
    \hline
    Pruebas                     &   &   &   &   &   & x &    \\
    \hline
    Training                    &   &   &   &   &   & x &    \\
    \hline
    Instalación                 &   &   &   &   &   &   & x  \\
    \hline
    Documentación               &   &   &   &   & x &   &    \\
    \hline
  \end{tabular}
  \caption{Responsabilidades de los miembros} \label{Tab:Respon}
\end{sidewaystable}

\chapter{Procesos de gestión}
\section{Objetivos y prioridades de gestión}
Las prioridades del proyecto son mejorar y ampliar el uso y la calidad de la
herramienta software. Los objetivos son los siguientes:
\begin{itemize}
\item Prestar apoyo en materia de las TI a todas las actividades relacionadas
  con la investigación, la docencia y la gestión.
\item Agilidad en las consultas de los planes de estudio.
\item Rapidez y fiabilidad en la recogida de datos para completar la
  herramienta software.
\item Entrega rápida de cara al comienzo del curso 2010-2011. La consulta de
  los planes de estudio tiene una gran demanda en el mes de septiembre debido
  a las matriculaciones de las distintas carreras.
\item Se mejora considerablemente la facilidad de trabajo de todos los
  usuarios que utilizan la herramienta.
\item Reducir el gasto en la contratación de personal para elaborar guías
  docentes. 
\end{itemize}

\section{Supuestos, dependencias y restricciones}
El proyecto depende de tres planteamientos principales:
\begin{itemize}
\item La herramienta software.
\item Pruebas del software utilizado para ver su efectividad.
\item Habilidad de los usuarios tanto administradores como consultores para
  utilizar la herramienta con facilidad.
\end{itemize}
Pueden surgir problemas con alguno de estos plantemientos, en cuyo caso, la
ejecución del proyecto se vería afectada de cara a los plazos y condiciones
de entrega.

Las suposiciones que se tienen en cuenta son las siguientes:
\begin{itemize}
\item Los usuarios no tienen la experiencia necesaria para manejar el SI, por
  lo cual, se necesita una adaptación al SI. 
\item Todos los usuarios tienen la habilidad necesaria para manejar
  adecuadamente el sistema, ya que el hecho de que un usuario no pudiese
  utilizar al 100\% la efectividad del SI, decrementaría considerablemente su
  eficacia y no podría en ningún momento satisfacer los requisitos exigidos.
\item El proyecto debe de estar íntegramente terminado antes del comienzo del
  curso 2010-2011.
\end{itemize}

\section{Gestión de riesgos}
En la tabla \ref{Tab:GestRi} se muestran los riesgos de mayor impacto del
proyecto. 
\begin{table}[!h]
  \centering
  \begin{tabular}{p{4.5cm}|c|c|b{2cm}<{\centering}}
    %\cline{2-4}
    & \textbf{Probabilidad} & \textbf{Impacto} & \textbf{Exposición al riesgo}\\
    \hline \hline
    Planificación optimista 
    & 0,70 & 7 & 0,90 \\ \hline
    Cambios de los requisitos durante la ejecución del proyecto 
    & 0,25 & 2 & 0,50 \\ \hline
    Valorar la calidad & 0,35 & 2 & 0,70 \\ \hline
    Diferencias entre administradores y usuarios & 0,2 & 2 & 0,40 \\ \hline
    Retrasos en la entrega de la herramienta & 0,20 & 2 & 0,50 \\ \hline
    Valorar la implantación de la herramienta & 0,25 & 3 & 0,6 \\ \hline
    Personal no competente & 0,20 & 3 & 0,80 \\ \hline
  \end{tabular}
  \caption{Gestión de Riesgos} \label{Tab:GestRi}
\end{table}

\section{Mecanismos de supervisión y control}
El grupo de trabajo y el analista realizan la planificación del proyecto
junto con el coordinador que se encarga de supervisar las tareas que realizan 
ambos.
Mediante las pruebas de la herramienta software, que realiza el usuario
experto, se realiza la supervisión del buen funcionamiento de ésta,
encontrando los posibles fallos y realizando los cambios oportunos. 
El coordinador lleva un control de las fechas de la entrega del proyecto.

\section{Plan del personal}
El número de personas requerido para llevar a cabo el proyecto son 10:
coordinador; grupo de trabajo que lo forman 4 personas; analista;
programador; secretario; usuario experto y operario servicio técnico. 
El proyecto consta de varias etapas en las que intervienen distinto tipo de
personal. En un principio se necesita el trabajo del coordinador, el grupo
de trabajo y el analista para realizar la planificación del proyecto. Se
especifican los requisitos y diseño del sistema. A partir de lo anterior, se
programa el software y se realizan las pruebas pertinentes que llevarán a
tener una herramienta eficiente. Por último, se instala la herramienta.

\chapter{Procesos técnicos}
\section{Métodos, herramientas y técnicas}
Los métodos, herramientas y técnicas usados por el GACGD comúnmente 
utilizados por los proyectos habituales de la UMA. El ciclo de vida básico es 
el de cascada. Tanto el análisis de requisitos como la programación se hacen 
de acuerdo a los métodos estructurados conocidos. Dado que el alcance de la 
UMA abarca más de una ciudad, se utiliza una WAN por sus posibilidades y 
porque cubre áreas muy amplias. Las herramientas utilizadas para la 
construcción del proyecto deben funcionar para la WAN y sus respectivos SSOO. 
El lenguaje de programación a utilizar es Python, ya que proporciona 
facilidades a entornos como el del proyecto. El sistema operativo elegido es 
Debian GNU/Linux Lenny.

\section{Documentación del software}
Ya que se sigue un ciclo de vida en cascada, la documentación final del 
proyecto esta formada por cada uno de los siguientes documentos, generados en 
cada una de las etapas del modelo:

\begin{description}
\item[Análisis] Catálogo de requisitos, definición del sistema y plan de
  pruebas.  
\item[Diseño] Especificación de construcción del SI.
\item[Implementación] Manual de usuario y programa de formación software. 
\item[Pruebas] Resultados de pruebas unitarias, de integración y del sistema.
\end{description}

\section{Funciones de soporte a proyectos}
El departamento de SI y los consultores proveen al proyecto de las debidas 
funciones de soporte. Estos se encargan de proponer las pruebas y sus 
procedimientos asociados. Estas actividades son las de validación y 
verificación. Además se encargan de aportar soporte técnico a aquellas áreas 
donde los miembros no tengan experiencia, éstas áreas son el análisis y diseño 
de la WAN y documentación del análisis de requerimientos.

\chapter{Paquetes de trabajo}
\section{Análisis}
\subsection{Definición del sistema}
Determinación del alcance del sistema, de la tecnología que se va a usar, los
estándares que se van a seguir para su  construcción teniendo en cuenta los
usuarios a quienes va destinado.
\begin{description}
\item[Duración estimada] 3 días.
\end{description}

\subsection{Establecimiento de requisitos}
Obtención, análisis y validación de los requisitos valiéndose de herramientas
como los diagramas de casos de uso.
\begin{description}
\item[Duración estimada] 6 días.
\end{description}

\subsection{Identificación de subsistemas}
Incluye la determinación de los distintos subsistemas y su posterior
integración.
\begin{description}
\item[Duración estimada] 2 días.
\end{description}

\subsection{Elaboración del modelo de datos}
\subsubsection{Elaboración del modelo conceptual y lógica de datos}
Identifica y define las entidades que quedan dentro del SI, posteriormente se
preparan las relaciones complejas y se eliminan redundancias y ambigüedades. 
\begin{description}
\item[Duración estimada] 2 días.
\end{description}

\subsubsection{Normalización}
Se revisa el modelo lógico de datos para eliminar redundancias e
inconsistencias en las entidades de datos. 
\begin{description}
\item[Duración estimada] 2 días.
\end{description}

\subsubsection{Especificación de necesidades de carga inicial}
Incluye las necesidades hardware y estimaciones de capacidades.
\begin{description}
\item[Duración estimada] 2 días.
\end{description}

\subsection{Elaboración del modelo de procesos}
Consiste en un análisis de las necesidades del usuario para establecer el
conjunto de procesos del SI. 
\begin{description}
\item[Duración estimada] 4 días.
\end{description}

\subsection{Definición de interfaz de usuario}
Aquí se definen las interfaces entre el sistema y el usuario: formatos de
pantallas, diálogos, e informes, principalmente. 
    \begin{description}
\item[Duración estimada] 2 días.
\end{description}

\subsection{Análisis de consistencia y especificación de requisitos}
Consiste en verificar la calidad técnica del modelo, cerciorándose de la
coherencia entre modelos y del cumplimiento de los requisitos. 
\begin{description}
\item[Duración estimada] 3 días.
\end{description}

\subsection{Especificación de plan de pruebas}
Incluye la definición del alcance y los requisitos del plan de pruebas.
\begin{description}
\item[Duración estimada] 3 días.
\end{description}

\subsection{Aprobación del análisis del SI}
Consiste en la presentación y posterior aceptación del análisis del SI.
\begin{description}
\item[Duración estimada] 1 día.
\end{description}

\section{Diseño}
\subsection{Definición de la arquitectura del sistema}
Se define la arquitectura general del SI, especificando las distintas
particiones físicas del mismo, la descomposición lógica en subsistemas de
diseño y la ubicación de cada subsistema en cada partición, así como la
especificación detallada de la infraestructura necesaria para dar soporte al
SI. 
\begin{description}
\item[Duración estimada] 4 días.
\end{description}

\subsection{Diseño de la arquitectura de soporte}
Se especifica la arquitectura de soporte, que comprende el diseño de los
subsistemas de soporte identificados en la actividad de Definición de la
Arquitectura del Sistema y la determinación de los mecanismos genéricos de
diseño.
\begin{description}
\item[Duración estimada] 2 días.
\end{description}

\subsection{Diseño de la arquitectura de módulos del sistema}
\subsubsection{Diseño de módulos del sistema}
Se realiza una descomposición modular de los subsistemas específicos
identificados en la tarea de identificación de subsistemas de diseño, que es
uno de los subprocesos contenidos dentro del proceso de definición de la
arquitectura del sistema. 
\begin{description}
\item[Duración estimada] 2 días.
\end{description}

\subsubsection{Diseño de comunicación entre módulos}
Se definen las interfaces entre los módulos de cada subsistema, entre
subsistemas y con el resto de los sistemas. 
\begin{description}
\item[Duración estimada] 2 días.
\end{description}

\subsubsection{Revisión de la interfaz de usuario}
Se realiza el diseño detallado de la interfaz de usuario a partir de la
especificación obtenida en el Análisis del Sistema de Información. 
\begin{description}
\item[Duración estimada] 1 día.
\end{description}

\subsection{Diseño físico de datos}
\subsubsection{Diseño del modelo físico de datos}
Se realiza el diseño del modelo físico de datos a partir del modelo lógico de
datos normalizado o del modelo de clases, en el caso de diseño orientado a
objetos. 
\begin{description}
\item[Duración estimada] 2 días.
\end{description}

\subsubsection{Especificación de los caminos de acceso a los datos}
Se determinan los caminos de acceso a los datos persistentes del sistema con
el fin de optimizar el rendimiento de los gestores de datos o sistemas de
ficheros y el consumo de recursos, así como disminuir los tiempos de
respuesta. 
\begin{description}
\item[Duración estimada] 1 día.
\end{description}

\subsubsection{Especificación de la distribución de datos}
Se determina el modelo de distribución de datos, teniendo en cuenta los
requisitos de diseño establecidos. 
\begin{description}
\item[Duración estimada] 1 día.
\end{description}

\subsection{Verificación y aceptación de la arquitectura del sistema}
Tiene como objetivo el garantizar la calidad de las especificaciones del
diseño del sistema de información y la viabilidad del mismo, como paso previo
a la generación de las especificaciones de construcción. 
\begin{description}
\item[Duración estimada] 2 días.
\end{description}

\subsection{Generación y especificación de construcción}
Se generan las especificaciones para la construcción del SI, a partir del
diseño detallado. Estas especificaciones definen la construcción del SI a
partir de las unidades básicas de construcción. 
\begin{description}
\item[Duración estimada] 3 días.
\end{description}

\subsection{Diseño de migración y carga incial de datos}
De acuerdo a la estructura física de los datos del nuevo sistema, se procede
a definir y diseñar en detalle los procedimientos y procesos necesarios para
realizar la migración. 
\begin{description}
\item[Duración estimada] 4 días.
\end{description}

\subsection{Especificación técnica del plan de prueba}
Se realiza la especificación de detalle del plan de pruebas del SI para cada
uno de los niveles de prueba establecidos en el Análisis del SI (unitarias,
de integración, del sistema, de implantación y de aceptación). 
\begin{description}
\item[Duración estimada] 2 días.
\end{description}

\subsection{Establecimiento de requisitos de implantación}
Se completa el catálogo de requisitos con aquellos relacionados con la
documentación que el usuario requiere para operar con el nuevo sistema y los
relativos a la propia implantación del sistema en el entorno de operación. 
\begin{description}
\item[Duración estimada] 3 días.
\end{description}

\subsection{Aprobación del diseño y SI}
Se realiza la presentación del diseño del SI al Comité de Dirección para la
aprobación final del mismo. 
\begin{description}
\item[Duración estimada] 1 día.
\end{description}

\section{Implementación}
\subsection{Preparación del entorno de generación y construcción}
El objetivo es asegurar la disponibilidad de todos los medios y facilidades
para que se pueda llevar a cabo la construcción del SI. 
\begin{description}
\item[Duración estimada] 5 días.
\end{description}

\subsection{Generación del código de los componentes y los procedimientos}
El objetivo es la codificación de los componentes del SI a
  partir de las especificaciones de construcción en el proceso de diseño del SI.
\begin{description}
\item[Duración estimada] 5 días.
\end{description}

\subsection{Elaboración del manual de usuario}
Elaboración de la documentación de usuario, tanto usuario final como de
explotación, de acuerdo a los requisitos recogidos en el catálogo de
requisitos. 
\begin{description}
\item[Duración estimada] 3 días.
\end{description}

\subsection{Definición de la formación de los usuarios finales}
Se establecen las necesidades de formación del usuario final, con el objetivo
de conseguir la explotación eficaz del nuevo sistema. 
\begin{description}
\item[Duración estimada] 4 días.
\end{description}

\subsection{Construcción de los componentes y procedimientos de 
carga inicial de datos} 
Codificación y prueba de los componentes y procedimientos de migración y
carga inicial de datos, a partir de las especificaciones recogidas en el plan
de migración y carga inicial de datos obtenido en el proceso de Diseño del
SI. 
\begin{description}
\item[Duración estimada] 4 días.
\end{description}

\section{Pruebas}
\subsection{Ejecución de las pruebas unitarias}
Se realizan las pruebas unitarias de cada uno de los compomentes del SI, una
vez codificados, con el objeto de comprobar que la estructura es correcta y
se ajusta a su funcionalidad. 
\begin{description}
\item[Duración estimada] 2 días.
\end{description}

\subsection{Ejecución de las pruebas de integración}
Verificar si los componentes o subsistemas interactúan correctamente a través
de sus interfaces, cubren la funcionalidad y se ajustan a los requisitos 
especificados. 
\begin{description}
\item[Duración estimada] 2 días.
\end{description}

\subsection{Ejecución de las pruebas del sistema}
Comprobación de la integración del sistema de información globalmente,
verificando el funcionamiento correcto de las interfaces entre los distintos
subsistemas que lo componen y con el resto de SI con los que se comunica. 
\begin{description}
\item[Duración estimada] 2 días.
\end{description}

\subsection{Aprobación del SI}
Se recopilan los productos de SI y se presentan al Jefe de Proyecto para su
aprobación.
\begin{description}
\item[Duración estimada] 1 día.
\end{description}

\bibliographystyle{plain} 
\bibliography{t2}

\end{document}
