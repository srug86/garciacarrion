% Clase
\documentclass[11pt,a4paper,spanish,twoside]{report}

% Órdenes auxiliares
\input{inc/includes.tex}

% Encabezado y pie de página
\encabezado

\setcounter{secnumdepth}{3}
\setcounter{tocdepth}{3}

\begin{document}

% Silabación extra
\input{inc/hyphenations.tex}

% Portada
\portada{Automatización Industrial}
{Trabajo:}{Automatización del proceso de embotellado en\\ J. García Carrión(Daimiel)}
{Sergio de la Rubia García-Carpintero\\Miguel Millán Sánchez-Grande}{30 de Abril de 2010}

% Licencia
\licencia{Sergio de la Rubia García-Carpintero, Miguel Millán Sánchez-Grande}

% Índices
\tableofcontents
% \listoffigures
% \listoftables

%% INICIO DEL DOCUMENTO %%%%%%%%%%%%%%%%%%%%%%%%%%%%%%%%%%%%%%%%%%%%%%%%%

\chapter{Introducción}
%soltar el rollo de xq la hemos elegido, en que consiste.
A la hora de elegir una empresa sobre la que realizar el trabajo, se buscaron
opciones cercanas geográficamente y de fácil acceso en nuestro entorno. Por
este motivo la opción elegida fue la de la empresa J.Garcia Ca\-rrión (a partir
de ahora JGC), ubicada en Daimiel; por tratarse de una instalación próxima 
a Ciudad Real y porque el hermano de uno de los componentes de nuestro grupo
trabaja en dicha planta.

JGC es el encargado de todos los productos Don Simón. 

JGC para desarrollar su plan estratégico cuenta con distintas bodegas y plantas
de proceso y envasado de última tecnología en distintas partes de la
geografía española, siendo una empresa pionera tanto a nivel nacional como
internacional:

\begin{itemize}
\item Jumilla (Murcia), origen de la empresa, con una capacidad de 400
  millones de litros. 
\item Daimiel (Ciudad Real), pionera en tecnología a nivel mundial, cuya alta
  producción de 800 millones de envases por año, la hace la más eficiente de
  Europa. Sobre la que se centra el trabajo.  
\item Gador (Almería), inaugurada en el año 2003, es una Planta especia\-li\-za\-da
  en tratamiento de vegetales (gazpacho, cremas de verduras na\-tu\-ra\-les y
  caldos), con una capacidad de 120 millones de Kilos/año. La tecnología de
  esta planta se ha proyectado con Investigación y Desarrollo propio, único
  en la elaboración de productos naturales vegetales. 
\item Huelva, planta única en el mundo por su agricultura integrada. 
\end{itemize}

\section{Historia breve}
La familia JGC siempre ha estado ligada a los viñedos y a la tradición
agrícola, tiene sus orígenes en el pueblo murciano de Jumilla. En 1890 la 
familia construyó una nueva bodega, con ciertas dimensiones para aquella
época, debido al gran auge de la exportación del vino de Jumilla a
Francia, esta exportación fue originada por un parásito llamado filoxera que
arrasó con los viñedos del país galo. Por tanto se toma el año 1890 como él de la
fundación de GARCIA-CARRIÓN.

La marca GARCIA-CARRIÓN, prosigue su andadura, creciendo poco a poco y
sobreponiéndose a distintas adversidades como la guerra civil. Aún así
comenzó a distribuirse por toda España. Al continuar aumentando la demanda, 
se decide construir una nueva bodega en las proximidades de Jumilla, e
instalar el primer tren de embotellado de alta capacidad.
  
 
Si por algo se ha caracterizado siempre la compañía ha sido por la innovación
y por arriesgarse a realizar cosas que otros pensaban imposibles, de
esta forma, a principios de los 80, con la implantación de las grandes
superficies en España, el envase más utilizado para el vino de mesa era la
botella de 1L. retornable, lo que exigía la posesión varias plantas de
envasado distribuidas por toda España para atender la demanda nacional. Con
el fin de buscar un envase no retornable, práctico, económico y de poco peso
la compañía optó por la tecnología brik.

El lanzamiento del nuevo envase coincidió con el primer anuncio en televisión
de la compañía, que será siempre recordado por la frase : ``Voy a comer con
Don Simón''.

\chapter{Objetivo y motivación del sistema}
La apertura de la planta de Daimiel coincide con la linea general de
desarrollo e innovación que sigue la compañía JGC desde su creación, como
se ha podido observar a través de su historia. Una primera planta fue
inaugurada hace ahora tres años, esta la planta cubría una necesidad creada
a partir del aumento de la demanda de sus productos. Era necesario una forma
de elaborar mucho más producto, de una forma más veloz y cómo no a un menor
precio. La alternativa de aumentar el imperio JGC sobre la geografía española
se planteaba como una alternativa cara de ejecutar, así que se optó por la
construcción de una sola planta en la que invertir más dinero para conseguir
un alto rendimiento.

Pero durante la primera producción de dicha planta un incendio acabo con
ella. Esta planta funcionaba con maquinas de la marca Sindel, y se encargaba
de la producción de tinto de verano. Esa misma producción tuvo que ser
finalizada en la planta antigua que la empresa posee en Jumilla.

Con el dinero que se consiguió con la liquidación del seguro unido a una
subvención de la Junta de Comunidades de Castilla la Mancha, se decidió
construir la planta que conocemos a día de hoy. Una de las punteras
tecnológicamente hablando, no sólo a nivel nacional, sino que también a nivel
internacional. Como posteriormente se explicará, las máquinas que desarrollan
la cadena de trabajo son de la marca alemana Krones.

\chapter{Descripción del proceso}
El proceso que se lleva a cabo en la fábrica de JGC situada en Daimiel y
sobre el cual se centra el esfuerzo de este trabajo consiste en embotellar
zumos, sangría o incluso tinto de verano, dependendiendo de las necesidades,
pero sobre todo tinto de verano. Es decir en general cualquier producto de
Don Simón que esté envasado en botella de plástico.

En la \emph{figura \ref{proceso}} se representa un mapa de todo el proceso
automatizado. Como se aprecia consta de seis etapas que se han de cumplir
secuencialmente, para que dado una preforma y el líquido que se quiere
embotellar consigamos un palet lleno de botellas.
\imagen{proceso.pdf}{12.5}{Proceso de embotellamiento}{proceso}

\begin{itemize}
\item La primera etapa consiste en a partir de las preformas obtener botellas de
  plástico. Una preforma es como se puede apreciar en la \emph{imagen  
  \ref{preforma}} una pequeña probeta de plástico, esta preforma es
  introducida en una máquina llamada \textbf{Sopladora}. A la salida de esta etapa las
  preformas son convertidas en botellas de plástico vacías.
\imagen{preforma-20.pdf}{4.5}{Preforma}{preforma}

\item En la siguiente etapa entra en juego no sólo la salida de la etapa
  anterior, si no que también el tinto. Ambos se compaginan en la
  \textbf{Llenadora/Taponadora} a la salida de esta etapa se tienen las
  botellas llenas y tapadas. Aquí no se abarca pero en esta etapa existe un
  inspector del nivel del tapón que se encarga de controlar que los tapones
  están correctamente colocados.

\item Ahora se pasa a etiquetar las botellas, se introducen en la
  \textbf{Etiquetadora} y a la salida se obtienen las botellas correctamente
  etiquetadas, esta etapa también consta de un inspector para confirmar que
  las etiquetas fueron colocadas de forma correcta. Posteriormente se
  profundizará más sobre ello.

\item Posteriormente las botellas son conducidas hacia la \textbf{Agrupadora}
  que como su nombre indica que encarga de agrupar las botellas y forrarlas
  en plástico de forma que su futura manipulación se tratará como un conjunto
  de botellas y no de forma individual.

\item Después los grupos de botellas pasan a otra máquina llamada
  \textbf{Paletizadora} que se encarga de colocar los grupos de botellas unos
  encima de otros, y todos ellos encima de un palé para facilitar su
  transporte.
\item Finalmente el palé de botellas pasa hacia la \textbf{Enfardadora}
  donde se forra con unas capas de plástico transparente para evitar pérdidas
  en el transporte o que el producto sufra algún destrozo.

\end{itemize}
Aparte de las diversas islas de automatización de las que se ha hablado, todo
el transporte de una a la siguiente también está automatizado mediante diversas
cintas transportadoras. Pero de todo esto se habla más a fondo en el capítulo
siguiente. 

\chapter{Explicación de las islas de automatización}
\section{Sopladora}
La primera de nuestras islas de automatización es la sopladora.

Esta planta dispone de una sopladora KRONES Volumetic, con una capacidad de 
trabajo de más de 14.000 botellas a la hora.

Las preformas se encuentran dispuestas en una fila a la entrada de un horno. 
Cada preforma entra al túnel y se calienta por radiación. Según el tipo de 
resina de con la que está hecha y la geometría que se quiere conseguir, se 
aplica una temperatura específica (unos 110ºC de media). Para controlar la 
temperatura que se aplica a cada preforma, el horno cuenta con un pirómetro.
Con la ayuda de un PID, la temperatura que se aplica a la preforma, se 
mantiene en todo momento constante, ya que de no ser así puede provocarle
deformaciones.

Una vez caliente, la preforma es conducida a un molde. El molde está compuesto 
por dos o tres piezas que se separan para permitir la entrada de la preforma. 
Una vez dentro, las piezas se unen y desciende la sopladora sobre la boquilla 
de la preforma. La sopladora realiza dos operaciones sobre la preforma:
\begin{itemize}
\item Primero, se realiza el estirado y presoplado. Este consiste en que una 
varilla se introduce en la preforma y la va estirando mientras la sopla hasta 
que le da la longitud final que tendrá la botella.
\item Después, se realiza el soplado moldeante. Este hace que la preforma se
hinche hasta tocar las paredes del molde. Por el interior del molde circula 
una corriente de agua (a una temperatura aproximada de 9ºC) que mantiene frías 
las paredes. Cuando el material caliente de la preforma toca la pared fría del 
molde, adquiere la forma de este y se enfría.
\end{itemize}
Para controlar los procesos de estirado y soplado, el sistema cuenta también 
con un controlador PID, que maneja el traductor de presión y la válvula 
modulante de presión.

En total, la sopladora consta de 8 moldes. Estos se encuentran dispuestos 
sobre una superficie giratoria que, mientras gira, va cogiendo las preformas 
que se le van sirviendo para meterlas en cada uno de los moldes. Una vez 
formada cada botella, aprovechando el giro del sistema, se vuelve a abrir el 
molde y va saliendo del sistema por el lado opuesto al que entró, hacia la 
siguiente isla de automatización, la llenadora.

\section{Llenadora/Taponadora}
Las botellas son transportadas por un carrusel hasta la llenadora.

La llenadora/taponadora de esta planta, es una KRONES Contiform, con una 
capacidad de trabajo de 12.000 a 14.000 botellas a la hora, dependiendo de la 
capacidad de las botellas que deben llenar.

Al llegar a este sistema, cada botella es enchufada por un caño. Esta máquina 
posee un total de 80 caños conectados a un depósito donde está alojado el 
líquido con el que vamos a llenar las botellas, en este caso tinto de verano.

Primero, cada caño hace una presurización de la botella, con el objetivo de 
equilibrar la presión existente entre la botella y el caño para que pueda caer 
el líquido en la botella y que no forme espuma.

Para conocer el volumen que contiene la botella en un instante dado, cada caño 
posee un caudalímetro. Como el llenado de la botella se produce mientras se 
transporta a la siguiente máquina del sistema, se debe configurar correctamente
la velocidad del carrusel para que le de tiempo a llenar la botella.

Este sistema, al igual que casi todos los sistemas de la planta, utiliza varias
fotocélulas de ultrasonidos para detectar el paso de las botellas, y 
codificadores rotatorios (o encoders) para calcular, mediante el conteo de 
pulsos, cuando deben actuar los actuadores. En este caso, detecta el paso de 
una botella y calcula el momento en que esta llegará al caño, para dar la orden
de abrir a la válvula de llenado de dicho caño.

Cuando la botella sale del carrusel de llenado entra en la taponadora. La 
taponadora detecta la entrada de una botella, abre una trampilla, deja caer el 
tapón y mediante un embrague, que realiza un par determinado, lo aprieta, 
tapando la botella.

A fin de controlar las botellas defectuosas, este sistema cuenta con dos 
inspectores:
\begin{description}
\item \textbf{Inspector de nivel}
Este primer inspector se encarga de controlar que la botella ha sido llenada 
hasta un nivel esperado. Las botellas que tienen un nivel defectuoso serán 
retiradas de la cadena. Este inspector detecta el nivel de la botella mediante 
rayos X y proporciona una salida entre 0 y 10V.

\item \textbf{Inspector de tapón}
Este inspector se encarga de controlar que la botella esté tapada 
correctamente. Para ello, se cuenta con dos fotocélulas. Estas fotocélulas se 
encuentran a la altura a la que debería ir la anilla del tapón. La anilla del 
tapón hace más grueso el cuello de la botella. Las fotocélulas se sitúan a una 
distancia algo menor al diámetro del cuello de la botella, de tal forma que si
ambas fotocélulas detectan a la botella a la vez es que lleva bien colocado el 
tapón. En cualquier otro caso se controlará la botella para ser retirada  de 
la cadena.
\end{description}

Como se dijo anteriormente, este sistema controla, mediante un encoder, las 
botellas defectuosas para retirarlas de la cadena. Las botellas son 
"literalmente'' empujadas fuera de la cadena, en su fase de transporte, 
mediante un expulsador (o pusher). Este sistema empuja a la botella fuera de 
la cinta transportadora. El encoder cuenta los pulsos que deben pasar desde 
que un inspector encuentra una botella defectuosa, hasta que realmente esta 
botella pasa delante de uno de estos pusher; es entonces cuando se da la orden 
de que este actúe.

\section{Etiquetadora}
Antes de llegar a la siguiente isla de automatización, encontramos una 
\textbf{mesa de acumulación}. Esto se debe a que, a partir de este nivel, cada
una de las máquinas que encontraremos es un 10\% más rápida que la anterior.
Por este motivo, para que la etiquetadora, nuestro siguiente subsistema a 
analizar, pueda trabajar con continuidad, es decir, sin que le falten los 
recursos que necesita para trabajar, necesitará contar con un número 
considerable de botellas en stock. Cuando se alcanza este nivel, se activan 
las cintas transportadoras que conducen a las botellas a la etiquetadora.
Para controlar el número de botellas almacenadas, la mesa de acumulación 
cuenta con un sensor que mide la presión que ejercen las botellas contra las
paredes de la mesa. Cuando se alcanza la presión programada, la cinta 
transportadora comienza a moverse. Por otro lado, cuando esta presión es nula,
el movimiento de la cinta cesará y de forma automática, también se parará
la etiquetadora.

Llegados a este punto, encontramos a un gran número de botellas agrupadas 
desplazándose hacia la etiquetadora, aunque, esta solo puede recibir las 
botellas enfiladas. Es por ello que necesitamos un mecanismo intermedio que 
las alinee. Para ello se cuenta con un \textbf{alineador}. Este sistema cuenta
con una serie de estrechas cintas transportadoras, colocadas unas al lado de
las otras, de forma paralela. Y cada cinta se desplaza a una velocidad 
superior que la que se encuentra inmediatamente a su izquierda. Pues bien, el
sistema funciona de la siguiente manera: las botellas llegan apiñadas por la 
cinta más ancha y más lenta hasta que se encuentra en su camino con un 
estrechamiento. En ese momento, las botellas que van primeras, son empujadas
a una cinta más estrecha y más rápida. Las botellas que han sido empujadas a
esta cinta, se encontrarán de nuevo con otro estrechamiento, cayendo de nuevo
a otra cinta más rápida que la anterior. El resultado será que, cuando se 
llega a la última cinta, todas las botellas que se encuentran en cabeza, están
alineadas.

Por fin, las botellas llegán a la \textbf{etiquetadora}. Esta etiquetadora es 
una KRONES Contiroll, y tiene una capacidad de trabajo de más de 14.000 
botellas a la hora.

Un tornillo sin fin va separando las botellas entrantes, para evitar que lo 
hagan todas seguidas. La botella se encuentra con un primer cilindro 
giratorio, llamado \emph{de estrella}. Este recoge la botella del tornillo sin
fin y se la sirve, al otro lado, a otro cilindro giratorio en estella mucho 
mayor. Este cilindro recoge la botella en una de sus muecas y la hace también 
girar sobre sí misma. Durante su camino por el cilindro, la botella se 
encuentra primero con el sistema de etiquetado. Este, está formado por una 
serie de rodillos que seguidamente comentaremos: tenemos dos rodillos que 
contienen dos bobinas de etiquetas, una que va suministrando las etiquetas y
la otra que está para cuando se termine el primer rollo. La conmutación entre
rollos se realiza automáticamente, de forma que no hace falta parar la 
máquina para cambiar de rollo. 


\section{Agrupadora}
A la salida de la etiquetadora encontramos otra mesa de acumulación, puesto 
que, como ocurría en el caso anterior, la siguiente isla de automatización, 
tiene la capacidad de trabajar un 10\% más rápido que la etiquetadora.

Esta isla de automatización es la \textbf{agrupadora}. La agrupadora de esta 
planta es una KRONES Variopac.

Antes de llegar a la máquina, las botellas son ordenadas en filas de dos y 
separadas mecánicamente en packs de cuatro. 

El objetivo de la agrupadora es plastificar lotes de cuatro botellas para
luego meterlos en palés.

La agrupadora consiste, básicamente, en una cinta transportadora bajo la que 
se encuentran varios rodillos móviles y varios fijos. Los rodillos fijos se 
encargan de tensar el rollo de plástico con el que se plastificarán los lotes.
Los rodillos móviles portan el extremo del rollo y se mueven describiendo 
arcos sobre los packs, dejándolos cubiertos.

Una vez cubiertos, los packs pasan por un \textbf{túnel de retractilado}, 
donde se calientan a una temperatura de unos 235ºC, haciendo que el plástico
envolvente se encoja y fije las botellas del pack.

\section{Paletizadora}


\section{Enfardadora}
El palé llega a la enfardadora a través de la misma cinta que lo saca de la 
paletizadora. 

La enfardadora tiene tres partes bien diferenciadas:
\begin{description}
\item \textbf{Mesa de rotación}
La mesa de rotación es una sección de la propia cinta transportadora que tiene 
la funcionalidad de girar en un plano paralelo al del suelo, pero sobre un eje 
perpendicular a este. Esta característica se aprovecha para, dado un rollo de 
plástico colocado cerca del palé, ir enrollando con el plástico el contenido 
del palé con varias capas, de arriba a abajo y viceversa, según las 
características del palé y de su contenido.
\item \textbf{Pisón}
Una vez colocado el palé encima de la mesa de rotación, se fija al suelo con 
ayuda del pilón. El pilón es una columna anclada al suelo que posee, en su 
parte superior, una plataforma móvil que desciende hasta aprisionar contra la
mesa de rotación al palé. Dicha plataforma gira junto con el giro de la mesa de
rotación. Además, el pisón tiene la funcionalidad de medir la altura del palé 
para decírselo al robot. Esto lo hace ayudado por una resistencia lineal
variable; el valor de la resistencia determinará la altura del palé.
\item \textbf{Robot}
La enfardadora posee además un robot. Este es el encargado de colocar el rollo 
de plástico de tal forma que, el movimiento de la mesa de rotación, vaya 
enrollando el plástico en el palé. El movimiento del brazo robótico posibilita 
forrar el palé de arriba a abajo con un simple movimiento ascendente o 
descendente del portador del rollo. Pero además, este modelo, también permite 
cambiar de rollo de enfardado sin tener que parar la máquina. El robot posee 
dos rollos de plástico; uno es el que se está utilizando para realizar el 
enfardado y el otro es el de repuesto, cuando se acaba el rollo, el propio 
robot desecha el rollo gastado y acude a su portarrollos a por un recambio para
poder continuar su labor. El operario por su parte puede cambiar en ese momento
el rollo vacío del portarrollos para cuando vuelva a gastarse el rollo actual.
\end{description}

La enfardadora manera una serie de variables de actuación según la naturaleza
del palé a enfardar. Así se podrá configurar la velocidad de rotación de la 
mesa y la tensión ejercida por el robot a la hora de colocar el rollo de 
plástico para el forrado del palé.

Los palés forrados continúan por la cinta transportadora contigua hasta el 
extremo de esta. Una vez allí, una de los numerosos vehículos con control 
automatizado se encargará de recoger el palé y llevarlo al almacen.

\chapter{Beneficios de la automatización}
Introducir nuevas tecnologías en una empresa tradicional como JGC siempre
resulta un proceso complicado y que no agrada a todo el mundo. Pero no se
puede negar que supuso un empujón muy importante tanto a nivel económico como
a nivel de producción para la empresa. Errónamente se asocia el concepto de
automatización con una solución para reducir la cantidad de empleados en
plantilla, bien esta empresa es un ejemplo de que esto no sucede, ya que al
mecanizar el proceso relacionado con el empaquetado y relleno de botellas
ayudo a la expansión de la empresa creando más puesto de trabajos, eso sí
distintos a los que había con anterioridad. Aparte del beneficio económico de
reducir el número de empleados de ese área hubo otros beneficios adicionales
mayores:

\begin{description}
\item \textbf{Aumento de la eficiencia}

Los costos de producción se redujeron drásticamente al aumentar las
unidades de producto fabricadas por unidad de tiempo. 

\item \textbf{Incremento del volumen de producción}

Al aumentar el número de unidades fabricadas por unidad de tiempo se aumentó
la cantidad de unidades producidas así como el número de clientes a los que
poder atender.

\item \textbf{Estandarización de los procesos}

De esta forma se logró que los productos tuvieran siempre las mismas
características, al hacer el proceso repetitivo y siguiendo los mismos
pasos. Se puede tener la certeza que dos vasos de cualquier producto Don
Simón van a tener siempre el mismo sabor, color, densidad, etc.

\item \textbf{Reducción de los problemas de calidad}

Consecuentemente con la estandarización de procesos se consiguió aumentar la
calidad, se eliminó cualquier error posible relacionado con un despiste
humano, ya fuera debido al cansancio o a una negligencia. 

\item \textbf{Aumento de la competitividad}

Todas estas mejoras, más y mejor producto en menos cantidad de
tiempo, tienen una clara repercusión en la competitividad de la empresa,
haciendo más fácil el cubrir diversas áreas de comercio. 

\item \textbf{Centralización de producción}

JGC es una de las empresas más grandes e importantes de España en el sector,
y como se ha hablado en la introducción consta solo de cuatro fabricas. Esto
es debido a la gran productividad asociada al uso de la automatización, no es
necesaria la construcción de mas sedes para cubrir las necesidad

\end{description}

Como demostración práctica de la utilidad de la automatización, la empresa
esta haciendo frente a la crisis que experimentamos en estos días con un
rebaja considerable en los precios de sus productos. Para realizar esto, JGC,
se está preparando para multiplicar la capacidad de producción 1,5 veces en
2014. Con el incremento de la producción pretende atender la demanda en
mercados internacionales para elevar su porcentaje de ventas al exterior
desde el 35\% actual al 60\%. Aún con la crisis la empresa tenía previsto
cerrar el ejercicio en curso con un beneficio antes de impuestos 22 millones
de euros, lo que supone un incremento del 22\% respecto al año anterior, y
elevar sus ventas un 12\%, hasta alcanzar los 650 millones de euros, en un
año en el que ha recortado sus precios un 20\% de media para hacer frente al
descenso provocado por la anteriormente nombrada crisis económica.

\bibliographystyle{plain} 
\bibliography{t2}

\end{document}
